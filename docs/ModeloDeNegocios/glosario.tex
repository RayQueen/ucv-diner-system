\documentclass{article}
\usepackage[spanish]{babel}
\usepackage[utf8]{inputenc}
\usepackage[margin=1in]{geometry}
\begin{document}
    \section*{Glosario de términos:}
    \medskip
    \begin{itemize}
        \item \textbf{Sistema del Comedor Universitario:} El software que sistematizará y automatizará la gestión del comedor universitario, facilitando el registro, asignación de turnos, control de consumo y gestión de insumos.
        \item \textbf{Estudiante:} Usuario del sistema que puede registrarse para acceder al comedor, consultar menús y acceder a turnos de comida.
        \item \textbf{Empleado:} Usuario del sistema que puede registrarse para acceder al comedor, consultar menús y acceder a turnos de comida.
        \item \textbf{Administrador:} Usuario del sistema que puede gestionar y visualizar los menús semanales, insumos disponibles, registro de consumo diario y generar reportes correspondientes a cada turno.
        \item \textbf{Turno:} Horario en el que se ofrece el servicio del comedor universitario a estudiantes y empleados.
        \item \textbf{Insumo:} Materia prima utilizada en la preparación de los alimentos del comedor universitario que debe ser tomada en cuenta al gestionar los menús semanales.
        \item \textbf{Consumo Diario:} Registro de los alimentos consumidos por los usuarios del comedor universitario en un día específico, gestionado por los administradores.
        \item \textbf{Reporte:} Documento generado por el administrador que permite analizar la demanda de un turno y planificar los recursos del comedor.
    \end{itemize}
\end{document}